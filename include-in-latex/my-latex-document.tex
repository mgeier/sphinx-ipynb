\RequirePackage{luatex85}
\documentclass[a4paper]{book}

% NB: This contains the "pxunit" config value (".75bp" by default):
\ifdefined\pdfpxdimen
   \let\sphinxpxdimen\pdfpxdimen\else\newdimen\sphinxpxdimen
\fi \sphinxpxdimen=.75bp\relax
\ifdefined\pdfimageresolution
    \pdfimageresolution= \numexpr \dimexpr1in\relax/\sphinxpxdimen\relax
\fi

\catcode`^^^^00a0\active\protected\def^^^^00a0{\leavevmode\nobreak\ }

\usepackage{fontspec}
\usepackage{mathpazo}
\linespread{1.05}  % see http://www.tug.dk/FontCatalogue/urwpalladio/
\setmainfont{TeX Gyre Pagella}[Numbers=OldStyle]
\setmonofont{Latin Modern Mono Light}[Numbers=Lining]

\usepackage{amsmath,amssymb,amstext}

\usepackage{polyglossia}
\setmainlanguage{english}

\usepackage[booktabs]{sphinx}
\sphinxsetup{
%verbatimwrapslines=false,
%verbatimhintsturnover=false,
noteBorderColor={HTML}{E0E0E0},
noteborder=0.5pt,
warningBorderColor={HTML}{E0E0E0},
warningborder=1.5pt,
warningBgColor={HTML}{FBFBFB},
}
\fvset{fontsize=\small}

\renewenvironment{sphinxnote}[1]
  {\begin{sphinxheavybox}\sphinxstrong{#1} }{\end{sphinxheavybox}}

\usepackage{nbsphinx}

\usepackage[
style=authoryear-comp,
backref=true,
]{biblatex}
% We need the original bibliography from the Sphinx project ...
\addbibresource{../doc/references.bib}
% ... and we can add our own if we want:
\addbibresource{my-own-references.bib}

% This should be the last package in the preamble:
\usepackage{hyperref}

\sphinxsetup{
InnerLinkColor={named}{black},
OuterLinkColor={named}{black},
}

\title{Including Sphinx-Generated Content in a \LaTeX\ Document}

\begin{document}

\maketitle

\tableofcontents

\chapter{A Hand-Written \LaTeX\ Chapter}

\section{References to Sphinx Content}

We can create references to the Sphinx content,
for example to section~\ref{code-cells:Code,-Output,-Streams}.


\section{Bibliography}

We can cite bibliography items from the Sphinx project:
\parencite{perez2011python}.
But we can also use our own bibliography file:
\parencite{knuth1986texbook}.

We only have to make sure that the bibliography keys are unique.

\section{Code Blocks}

We can use the \texttt{sphinxVerbatim} environment
to get code blocks like those generated by Sphinx:

\begin{sphinxVerbatim}
some code
\end{sphinxVerbatim}

\begin{sphinxVerbatim}[commandchars=\\\{\}]
some code with \emph{markup}
\end{sphinxVerbatim}

With a little additional work
we can even use the same syntax highlighting as Sphinx uses:

See \texttt{pygmentize.py}:

\input{pygmentize.py}

\dots and \texttt{latexmkrc}:

\input{latexmkrc}

\chapter{A Chapter Containing a Sphinx Project}

Depending on whether the last element of \texttt{latex\_documents}
(\url{https://www.sphinx-doc.org/en/master/usage/configuration.html#confval-latex_documents})
is \texttt{False} or \texttt{True},
the content of the main source file (without the main heading)
is shown here or not, respectively:

% NB: The name of the input file is defined with latex_documents in conf.py:
\input{_build/nbsphinx-chapter}

\printbibliography[title=References,heading=bibintoc]

\end{document}
